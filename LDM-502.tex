\documentclass[DM,lsstdraft,toc]{lsstdoc}
\usepackage{booktabs}
\bibliographystyle{gaia_aa}

\title[Measurement and Verification of KPMs]{The Measurement and Verification of DM Key Performance Metrics}
\author{David Nidever and Frossie Economou}
\date{May 6, 2016}
\setDocRef{LDM-502}
\setDocRevision{1.0b}
\setDocStatus{draft}

%
%   Revision history
%
% OLDEST FIRST: VERSION, DATE, DESCRIPTION, OWNER NAME
\setDocChangeRecord{%
\addtohist{1}{2016-05-06}{Initial version}{David Nidever}
}

\begin{document}

\maketitle

\section{Introduction}\label{introduction}

The Key Performance Metrics (KPMs) are the main criteria by which the
performance of the Data Management (DM) software system are evaluated
quantitatively. Many of these metrics are initially discussed in the
Science Requirements Document (\SRD, \citell{LPM-17}) but detailed further in the
LSST System Requirements (\LSR; \citell{LSE-29}) and the Observatory System
Specifications (\OSS; \citell{LSE-30}). They can be grouped into six main
categories: photometric, astrometric, shape, transient, computational
and system stability metrics.

It is important to note at the outset that these metrics are meant to
test to performance of the software and not the data itself. Therefore,
high quality datasets need to be used for these tests otherwise they
will only reveal the flaws in the data and not the software.

In the following, RSS stands for Root of Sum of Squares or adding all of
the error terms in quadrature.

IT IS CURRENLTY NOT CLEAR WHAT TYPES OF PHOTOMETRIC MAGNITUDES TO USE
FOR THE PHOTOMETRIC KPMS. SINGLE-FRAME MEASUREMENT, COADDS,
FORCED-PHOTOMETRY, MULTI-FIT, PSF, OR APERTURE?

\section{Photometric Metrics}\label{photometric-metrics}

There are several metrics on the precision (i.e. repeatability) and
accuracy of the photometry.

\subsection{Photometric Repeatability}\label{photometric-repeatability}

\subsubsection{Calibration Processing
Performance}\label{calibration-processing-performance}

Reference Document: \citell{LSE-30}; OSS-REQ-0275

\textbf{Discussion}: The LSST System shall photometrically calibrate raw
image data such that the data processing contributes no more than the
allocations of procCalRep for repeatability. These allocations include
effects from calibration algorithms, errors and noise in producing the
necessary calibration data products, as well as errors and uncertainties
in any reference catalogs used in the calibration process.

\begin{longtable}[]{@{}p{0.4\textwidth}llll@{}}
\toprule
Description & Value & Unit & Name & JIRA ID\tabularnewline
\midrule
\endhead
The maximum allowed RSS contribution to the overall photometric
repeatability of bright isolated point sources caused by errors
introduced in the data processing pipelines. & 0.003 & ABmag &
procCalRep & DLP-307\tabularnewline
\bottomrule
\end{longtable}

\textbf{Verification Owner:} SQuaRE

\textbf{Verification Technique:} Two visits (exposures) of the same
field with the same telescope/filter will be processed through the DM
science pipelines including calibration. The sources will then be
matched between the two visits and the RMS of the photometric
differences of bright stars computed.

\textbf{Verification Code:} Software exists to perform this calculation
using data processed through the first step of the science pipeline
(single frame processing with \texttt{processCcd}) called \htmladdnormallinkfoot{\texttt{validate\_drp}}{https://github.com/lsst/validate\_drp}. However, \texttt{validate\_drp}
does not include the calibration processing which are covered by this
KPM. Therefore, a modified version of this software will need to be
written that covers a larger portion of the DM data processing.

\textbf{Verification Implementation:} TBD

\textbf{Verification Data:} The KPM can be testing using precursor
datasets. The recommended dateset is the Hyper Suprime-Cam (HSC)
engineering data of the COSMOS field.

\subsubsection{Photometric Repeatability
Performance}\label{photometric-repeatability-performance}

Reference Document: \citell{LSE-29}; LSR-REQ-0093

\textbf{Discussion:} The specifications for photometric repeatability
applies to the cataloged LSST magnitudes,
m\textsuperscript{std}(catalog) (see \SRD eq. 8), for appropriately
chosen main sequence stars (e.g. non-variable stars color-selected from
the main stellar locus).

\begin{longtable}[]{@{}p{0.4\textwidth}llll@{}}
\toprule
Description & Value & Unit & Name & JIRA ID\tabularnewline
\midrule
\endhead
The RMS photometric repeatability of bright non-saturated unresolved
point sources in the \textbf{g}, \textbf{r}, and \textbf{i} filters. & 5
& milli-Mag & PA1gri & DLP-315\tabularnewline
The RMS photometric repeatability of bright non-saturated unresolved
point sources in the \textbf{u}, \textbf{z}, and \textbf{y} filters. &
7.5 & milli-Mag & PA1uzy & DLP-316\tabularnewline
\bottomrule
\end{longtable}

\textbf{Verification Owner:} SQuaRE

\textbf{Verification Technique:} The technique is essentially the same
as that used to compute the ``procCalRep'' KPM.

\textbf{Verification Code:} The same software as used for ``procCalRep''
KPM can be used here.

\textbf{Verification Implementation:} TBD

\textbf{Verification Data:} The same data as used for ``procCalRep'' KPM
can be used here.

\subsection{Photometric Spatial
Uniformity}\label{photometric-spatial-uniformity}

Reference Document: \citell{LSE-29}; LSR-REQ-0093. See also \SRD pg. 21.

\textbf{Discussion:} The distribution width (rms) of the internal
photometric zero-point error (the system stability across the sky) will
not exceed PA3 millimag. Use same source selection as for PA1gri.

\begin{longtable}[]{@{}p{0.4\textwidth}llll@{}}
\toprule
Description & Value & Unit & Name & JIRA ID\tabularnewline
\midrule
\endhead
RMS width of internal photometric zero-point (precision of system
uniformity across the sky) for u-band. & 20 & milli-Mag & PA3u &
DLP-317\tabularnewline
RMS width of internal photometric zero-point (precision of system
uniformity across the sky) in the g-band. & 10 & milli-Mag & PA3g &
DLP-318\tabularnewline
RMS width of internal photometric zero-point (precision of system
uniformity across the sky) in the y-band. & 10 & milli-Mag & PA3y &
DLP-319\tabularnewline
\bottomrule
\end{longtable}

\textbf{Verification Owner:} SQuaRE

\textbf{Verification Technique:} Checking the spatial uniformity of
ground-based photometric zero-points is challenging because there is no
hyper-accurate photometric dataset that covers the entire southern sky
(at least to the level that LSST needs). Therefore, the recommendation
is to use space-based data from the Gaia mission that should be accurate
across the sky to the few milli-mag level. Since Gaia uses a different
photometric system from LSST, photometric transformation equations from
Gaia's G-band to the LSST bands will need to be derived (one set per
filter). The Gaia G-band photometry, properly transformed, will then be
compared to ground-based photometry in various parts of the sky. To
improve the comparison only bright, blue, non-variable main-sequence
stars will be used. The median difference (zero-point error) in each
patch of sky will be computed and then the RMS computed over many
patches of sky. Note that if the Gaia dataset are used to calibrate the
LSST or pre-cursor data, then a separate test dataset will need to be
used such as the HST white dwarf standard stars.

\textbf{Verification Code:} Software will need to be written to ingest
the Gaia data and to determine and apply the transformation equations
from the ground-base photometric system to the Gaia G-band. Next,
software will be needed to select the necessary sample of bright, blue,
non-variable main-sequence stars in various regions across the sky,
compare the photometry to the Gaia and calculate the final metric of the
RMS of the differences in each band.

\textbf{Verification Implementation:} TBD

\textbf{Verification Data:} The Gaia G-band dataset (available in 2017)
or the HST white dwarf standard star data will be used as the reference.
The test data should be comprised of many fields spread across the sky
using a modern multi-chip imager (e.g. HSC, DECam, CFHT). A subset of
the HSC or DES survey data could be used for this purpose.

\subsection{Color Zero-point Accuracy}\label{color-zero-point-accuracy}

Reference Document: \citell{LSE-29}; LSR-REQ-0093, also see \SRD pg. 22

\textbf{Discussion:} The absolute band-to-band zero-point
transformations (color zero-points, e.g. for constructing the spectral
energy distribution, SED) for main-sequence stars must be known with an
accuracy of PA5 millimag. These requirements are primarily driven by the
desired accuracy of photometric redshift estimates. Note that an overall
stable gray error in the absolute calibration of the system does not
have an impact on the above requirements. Such an error is specified by
PA6. The same source selection as for PA1gri should be used.

\begin{longtable}[]{@{}p{0.4\textwidth}llll@{}}
\toprule
Description & Value & Unit & Name & JIRA ID\tabularnewline
\midrule
\endhead
Accuracy of absolute band-to-band color zero-point for all colors
constructed from any filter pair, excluding the u-band. & 5 & milli-Mag
& PA5 & DLP-320\tabularnewline
Accuracy of absolute band-to-band color zero-point for colors
constructed using the u-band. & 10 & milli-Mag & PA5u &
DLP-321\tabularnewline
\bottomrule
\end{longtable}

\textbf{Verification Owner:} SQuaRE

\textbf{Verification Technique:} This requirement is on the color of
stars. I can be tested by using spectrophotometric standard stars which
have well-determined space-based spectra. These spectra will need to be
convolved with the ground-based filter throughput curves to produce the
synthetic magnitudes and colors. The synthetic colors of a number of
standard stars will be compared to the calibrated ground-based colors
and the median difference measured.

\textbf{Verification Code}: The calibrated ground-based colors for the
calibrators will be compared to the synthetic colors of the standard
stars. Simple software will be needed to ``convolve'' the
spectrophotometric spectrum with the filter response curves to create
the synthetic photometry.

\textbf{Verification Implementation:} TBD

\textbf{Verification Data:} The HST standard star flux calibrators can
be used as the reference data\footnote{\url{http://www.stsci.edu/hst/observatory/crds/calspec.html}}. The test data
should be comprised of many fields spread across the sky (in the regions
where the calibrators are located) using a modern multi-chip imager
(e.g. SDSS, HSC, DECam, CFHT).

\subsection{Absolute Photometric
Accuracy}\label{absolute-photometric-accuracy}

Reference Document: \citell{LSE-29}; LSR-REQ-0093. See also \SRD, pg. 22.

\textbf{Discussion:} The LSST photometric system must transform to a
physical scale (e.g. AB magnitude scale) with an accuracy of PA6
millimag. The requirements are driven by the accuracy of absolute
determination of quantities such as luminosity and asteroid size for
objects with well determined distances. Note that the internal
band-to-band transformations are required to be much more ac- curate as
they may be calibrated and controlled by other means, and are not
sensitive to errors in overall flux scale of photometric calibrators.
Use same source selection as for PA1gri.

\begin{longtable}[]{@{}p{0.4\textwidth}llll@{}}
\toprule
Description & Value & Unit & Name & JIRA ID\tabularnewline
\midrule
\endhead
Accuracy of the transformation of the internal LSST photometry to a
physical scale (e.g. AB magnitudes). & 10 & milli-Mag & PA6 &
DLP-322\tabularnewline
\bottomrule
\end{longtable}

\textbf{Verification Owner:} SQuaRE

\textbf{Verification Technique:} To accurately perform absolute flux
calibration of ground-based photometry is quite challenging. Normally,
spectrophotometric standard stars, such as white dwarfs, are used for
this purpose. There is currently an effort underway, led by A. Saha and
C. Stubbs, to calibrate many hot DA white dwarfs (to the few millimag
level using HST photometry, Gemini spectroscopy and precise spectral
models) for the purpose of absolute flux calibration for surveys like
LSST. These standards will likely set the absolute flux calibration of
the LSST photometric system after an ubercal calibration has been
applied to the data. To test the accuracy of this calibration one would
want to use a subset of the calibrators to perform the calibration and
use the rest as the reference to test against.

\textbf{Verification Code:} The software needed is fairly
straightforward. The ground-based photometry for the calibrators
\emph{not} used in the calibration will be compared to the flux
calibrated spectrum of the star ``convolved'' with the filter response
function (which itself must be quite accurate).

\textbf{Verification Implementation:} TBD

\textbf{Verification Data:} Until the data for the new DA white dwarf
calibrators become available the existing HST standard star flux
calibrators can be used as the reference\footnote{\url{http://www.stsci.edu/hst/observatory/crds/calspec.html}}.\protect\hypertarget{_Toc301352570}{}{}
The test data should be comprised of many fields spread across the sky
(in the regions where the calibrators are located) using a modern
multi-chip imager (e.g. SDSS, HSC, DECam, CFHT). As this requirement is
also a constraint on the relative ubercal calibration across the sky, it
would be useful to use a large dataset like SDSS, Pan-STARRS1, DES or
the HSC survey to assess this metric.

\section{Astrometric Metrics}\label{astrometric-metrics}

There are relative and absolute requirements on the astrometry produced
by the LSST DM stack.

\subsection{Relative Astrometry}\label{relative-astrometry}

Reference Document: \citell{LSE-29}; LSR-REQ-0094. See also \SRD, pg. 23-24.

\textbf{Discussion:} This KPM concerns the astrometric quality of single
visit exposures. The rms of the astrometric distance distribution for
stellar pairs with separation of D arcmin (repeatability) will not
exceed AMx milliarcsec (median distribution for a large number of
sources). The three selected characteristic distances reflect the size
of an individual sensor, a raft, and the camera. The required median
astrometric precision is driven by the desire to achieve a proper motion
accuracy of 0.2 mas/yr and parallax accuracy of 1.0 mas over the course
of the survey. These two requirements correspond to relative astrometric
precision for a single image of 10 mas (per coordinate).

\begin{longtable}[]{@{}p{0.4\textwidth}llll@{}}
\toprule
Description & Value & Unit & Name & JIRA ID\tabularnewline
\midrule
\endhead
Median relative astrometric measurement error on 5 arcminute scales. &
\textless{}10 & milli-Arcsec & AM1 & DLP-310\tabularnewline
Median relative astrometric measurement error on 20 arcminute scales. &
\textless{}10 & milli-Arcsec & AM2 & DLP-311\tabularnewline
Median relative astrometric measurement error on 200 arcminute scales. &
\textless{}15 & milli-Arcsec & AM3 & DLP-312\tabularnewline
RMS difference between separations measured in the r-band and those
measured in any other filter. & 10 & milli-Mag & AB1 &
DLP-313\tabularnewline
\bottomrule
\end{longtable}

\textbf{Verification Owner:} SQuaRE

\textbf{Verification Technique:} LSST needs to have good relative
astrometry not just locally but over various scales on the sky. This can
be checked by selecting pairs of stars, measuring their distances many
times (over many visits), and determining how much the angular distance
varies around its mean value. This exercise is repeated for pairs of
stars separated on different angular scales. For AB1, the angular
distances of pairs of stars are compared between one band and the r-band
and the RMS of the differences determined.

\textbf{Verification Code:} Software will be needed to match sources
across different visits. Second, stars should be selected at random
across the field and their ``partner'' star with a separation close to
the angular scale that is being tested (5, 20, 200 arcmin). Then, the
angular distance for each pair should be measured for each visit and the
RMS determined about the mean. Finally, a median is taken of the
distribution of RMS values. For AB1, the angular distance is compared
between one band and the r-band and the RMS of the differences computed
(only one visits is needed for each band for this test).

\textbf{Verification Implementation:} TBD

\textbf{Verification Data:} The test data should be comprised of many
visits in a small region of the sky using a modern multi-chip imager
(e.g. HSC, DECam, CFHT). The HSC engineering data on the COSMOS field,
or a subset of the HSC or DES survey data could be used for this
purpose.

\subsection{Absolute Astrometry}\label{absolute-astrometry}

Reference Document: \citell{LSE-29}; LSR-REQ-0094. See also \SRD, pg.24.

\textbf{Discussion:} The LSST astrometric system must transform to an
external system (e.g. ICRF extension) with the median accuracy of AA1
milliarcsec.
The accuracy of absolute astrometry is driven by the
linkage and orbital computations for solar system objects. A somewhat
weaker constraint is also placed by the need for positional
cross-correlation with external catalogs. Note that the delivered
absolute astrometric accuracy may be fundamentally limited by the
accuracy of astrometric standard catalogs.

\begin{longtable}[]{@{}p{0.4\textwidth}llll@{}}
\toprule
Description & Value & Unit & Name & JIRA ID\tabularnewline
\midrule
\endhead
Median error in absolute position for each axis, RA \& DEC, shall be
less than AA1. & 50 & milli-Arcsec & AA1 & DLP-309\tabularnewline
\bottomrule
\end{longtable}

\textbf{Verification Owner:} SQuaRE

\textbf{Verification Technique:} The astrometric positions of stable
sources (e.g. bright quasars) will be compared between
astrometrically-calibrated LSST precursor datas and an astrometrically
accurate catalog such as Gaia.

\textbf{Verification Code:} The software will select stable sources from
a catalog of quasars. The positions will be compared to a standard
catalog, such as Gaia, and the median difference of the positions
computed.

\textbf{Verification Implementation:} TBD

\textbf{Verification Data:} The test data should be cover a large
portion of the sky to cover enough quasars. The SDSS, DES or HSC survey
datasets could be used for this purpose.

\section{Shape Metrics}\label{shape-metrics}

The measurements of shapes of galaxies are important for weak lensing
and constraining dark energy.

\subsection{Residual PSF Ellipticity
Corrections}\label{residual-psf-ellipticity-corrections}

Reference Document: \citell{LSE-29}; LSR-REQ-0097. See also \SRD, pg. 30-32.

\textbf{Discussion:} Using the full survey data, the E1 and E2 (see \SRD
for definitions) distributions averaged over an arbitrary FOV shall have
medians less than TE1 for theta \textasciitilde{} 1 arcmin, and less
than TE3 for theta \textless{} 5 arcmin. No more than TEF \% of images
shall have these medians for E1 and E2 larger than TE2 for theta
\textasciitilde{} 1 arcmin, or larger than TE4 for theta \textless{} 5
arcmin.

The requirements specified here require the full survey data set to
exist before they can be met. Thus these are intended to ensure that the
LSST system design enables that these requirements can be met after the
10- year survey. Prior to survey start, they will be verified to the
extent possible using simulations incorporating the as- built telescope
and camera performance characteristics.

\begin{longtable}[]{@{}p{0.4\textwidth}llll@{}}
\toprule
Description & Value & Unit & Name & JIRA ID\tabularnewline
\midrule
\endhead
Median residual PSF ellipticity correlations averaged over an arbitrary
field of view for separations less than 1 arcmin shall be no greater
than & 2.0e-5 & -\/- & TE1 & DLP-290\tabularnewline
Median residual PSF ellipticity correlations averaged over an arbitrary
field of view for separations between 1 and 5 arcmin shall be no greater
than & 1.0e-7 & -\/- & TE2 & DLP-308\tabularnewline
\bottomrule
\end{longtable}

\textbf{Verification Owner:} SQuaRE

\textbf{Verification Technique:} The mean correlation of PSF ellipticity
residuals needs to be measured between pairs of bright stars. The
ellipticity is measured for bright stars and the low-order dependence of
the ellipticity on the position in the focal plane is removed (creating
ellipticity residuals). The auto and cross-correlation functions of the
ellipticity residuals averaging over pairs of stars at a given angular
separation are then computed. An important component of the
ellipticities are the contributions from the optics. Therefore, this
needs to be tested on simulated LSST data.

\textbf{Verification Code:} Elliptical PSF models need to be fit to the
bright stars and PCA fit performed on the spatial dependence of the
ellipticity values to produce the ellipticity residuals. Pairs of stars
on certain angular scales should be selected and the auto and
cross-correlation functions of the ellipticity residuals measured.

\textbf{Verification Implementation:} TBD

\textbf{Verification Data:} Simulated LSST data of a small patch of sky
will be required to test this KPM. The Twinkles simulation data might be
sufficient for this test.

\section{Transient \& Moving Object
Metrics}\label{transient-moving-object-metrics}

The detection and characterization of transients, including solar system
objects, is an important component of LSST.

\subsection{Moving Object Linkage
Efficiency}\label{moving-object-linkage-efficiency}

Reference Document: \citell{LSE-30}; OSS-REQ-0159

\textbf{Discussion:} Valid identification and orbits shall be determined
for at least a fraction orbitCompleteness of Solar System objects which
are detected orbitObservations (2) times in orbitObservationInterval (3)
days at a level orbitObservationThreshold (5) sigma or more above the
single frame background.

Valid identification means that detections of the same Solar System
objects have been correctly associated as such. The verification method
for this requirement will be comparison with simulated inputs.

\begin{longtable}[]{@{}p{0.4\textwidth}llll@{}}
\toprule
Description & Value & Unit & Name & JIRA ID\tabularnewline
\midrule
\endhead
Minimum fraction of Solar System objects meeting reference criteria for
which valid orbits shall be determined. & 95\% & Percent &
orbitCompleteness & DLP-323\tabularnewline
\bottomrule
\end{longtable}

\textbf{Verification Owner:} SQuaRE, with help from University of
Washington initially

\textbf{Verification Technique:} The LSST reduction and MOPS software
will be run on simulated images that include solar system objects and
the completeness of the recovered objects measured.

\textbf{Verification Code:} The simulated data for several nights will
need to be processed through the Level 1, image differencing, pipeline
to detect the transients. These detections will then be fed to the MOPS
software to perform the linkages. Separate software will then compare
these linkages to the actual simulated orbits.

\textbf{Verification Implementation:} TBD

\textbf{Verification Data:} Simulated LSST images with solar system
objects for several nights are required to perform this test. These do
not exist yet and should be requested from the simulation team. A subset
of the data needed to test the MOPS spuriousness metric can be used
here.

\subsection{Spuriousness Metric Efficiency --
Transients}\label{spuriousness-metric-efficiency-transients}

Reference Document: \citell{LSE-30}; OSS-REQ-0353

\textbf{Discussion:} There shall exist a spuriousness threshold T for
which the completeness and purity of selected difference sources are
higher than transCompletenessMin and transPurityMin, respectively, at
the SNR detection threshold transSampleSNR (6). This requirement is to
be interpreted as an average over the entire survey.

This specification captures representative completeness and purity rates
supportive of time-domain science cases. Note that these are rates
determined only using the spuriousness metric cut; it is likely the
end-users will perform further classification steps to increase the
purity of their samples, depending on their particular science case.

This specification will be tested using simulations, by insertion and
recovery of artificial sources, and comparisons to ground truth where
known (i.e., asteroids, known variable stars, known variable quasars,
etc).

\begin{longtable}[]{@{}p{0.4\textwidth}llll@{}}
\toprule
Description & Value & Unit & Name & JIRA ID\tabularnewline
\midrule
\endhead
Minimum average completeness for transient science. & 90 & Percent &
transCompletenessMin & DLP-324\tabularnewline
Minimum average purity for transient science. & 95 & Percent &
transPurityMin & DLP-325\tabularnewline
\bottomrule
\end{longtable}

\textbf{Verification Owner:} SQuaRE, with help from University of
Washington initially

\textbf{Verification Technique:} The LSST reduction and image
differencing software will be run on simulated images with transients
(including solar system objects) and the completeness and purity of the
recovered transients measured.

\textbf{Verification Code:} The simulated data for several nights will
need to be processed through the Level 1, image differencing, pipeline
to detect the transients. These detections will then be compared to the
input transients to ascertain the completeness and purity.

\textbf{Verification Implementation:} TBD

\textbf{Verification Data:} Simulated data are needed to test this
requirement. As this metric is meant to be an average over the entire
survey, the simulated data should be for a representative patch of sky
(or many of them) and include visits for the entire survey with
realistic weather and must include the various types of transients. The
Twinkles data might be approaching what is needed for this, but it is
likely that more types of transients are needed and multiple fields on
the sky.

\subsection{Spuriousness Metric Efficiency -
MOPS}\label{spuriousness-metric-efficiency---mops}

Reference Document: \citell{LSE-30}; OSS-REQ-0354

\textbf{Discussion:} There shall exist a spuriousness threshold T for
which the completeness and purity of difference sources are higher than
mopsCompletenessMin and mopsPurityMin, respectively, at the SNR
detection threshold orbitObservationThreshold (5). This requirement is
intended to be interpreted as an average for any one month of observing.

This specification captures representative completeness and purity rates
needed to enable successful identification and linking of observed Solar
System objects. In particular, the need to have a Solar System object
repeatedly detected orbitObservation (2) times in
orbitObservationInterval (3) days strongly prefers high completeness,
even at the expense of purity.

This specification will be tested using simulations, by insertion and
recovery of artificial sources, and comparisons to ground truth where
known (i.e., asteroids, known variable stars, known variable quasars,
etc).

\begin{longtable}[]{@{}p{0.4\textwidth}llll@{}}
\toprule
Description & Value & Unit & Name & JIRA ID\tabularnewline
\midrule
\endhead
Minimum average completeness for Solar System object discovery & 99 &
Percent & mopsCompletenessMin & DLP-326\tabularnewline
Minimum average purity for Solar System object discovery & 50 & Percent
& mopsPurityMin & DLP-327\tabularnewline
\bottomrule
\end{longtable}

\textbf{Verification Owner:} SQuaRE, with help from University of
Washington initially

\textbf{Verification Technique:} The verification of this metric
proceeds very similarly to the Moving Object Linkage Efficiency metric.
The LSST reduction and MOPS software will be run on simulated images
that include solar system objects and the completeness and purity of the
recovered objects over month-long time spans are measured.

\textbf{Verification Code:} The simulated data for many months (maybe
\textasciitilde{}1 year) will need to be processed through the Level 1,
image differencing, pipeline to detect the transients. These detections
will then be fed to the MOPS software to perform the linkages and
compute orbits. Separate software will then compare the solar system
objects to those injected to determine the completeness and purity.

\textbf{Verification Implementation:} TBD

\textbf{Verification Data:} Simulated full focal-plane LSST images
including solar system objects for many months (maybe \textasciitilde{}1
year) are required to perform this test. These do not exist yet and
should be requested from the simulation team.

\section{Computational Performance
Metrics}\label{computational-performance-metrics}

There are various requirements on the computational performance of the
DM stack and hardware.

\subsection{Computational Budgets}\label{computational-budgets}

\textbf{Reference Document:} \citell{LDM-138}\textbar{}Output\textbar{}C4, F4 and
\citell{LSE-81}\textbar{}G263. Also see \SRD (pg.33-34) and \citell{LDM-140}.

\textbf{Discussion:} Data on likely optical transients will be released
with a latency of at most OTT1 (1) minutes. This includes the time to
transfer the data from Chile to the US, the computational time to
process the data and any overheads (e.g. I/O). The AP computational
budget is the amount of computational processing power the software
needs to meet the OTT1 with a reasonable amount of overhead. In a
similar vein, the DRP computational budget is the amount of
computational processing power the software needs to process the data in
1 year (6 months for the first year) with reasonable overheads.

\begin{longtable}[]{@{}p{0.4\textwidth}llp{1.5in}l@{}}
\toprule
Description & Value & Unit & Name & JIRA ID\tabularnewline
\midrule
\endhead
The minimum latency for releasing data on optical transients.
Performance of all AP pipelines, middleware and infrastructure. & 60 &
seconds & OTT1 & DLP-328\tabularnewline
Alert Production computational budget. Performance of all AP pipelines
and middleware. & 39 & TFLOPS & AP computational budget &
DLP-329\tabularnewline
Data Release Production computational budget (DR1). Performance of all
DRP+AP pipelines. & 108 & TFLOPS & DRP computational budget (DR1) &
DLP-314\tabularnewline
\bottomrule
\end{longtable}

\textbf{Verification Owner:} NCSA

\textbf{Verification Technique:} The AP and DRP computational budgets
are more straightforward to measure. A decent amount of simulated LSST
data should be processed through the respective pipeline and the total
number of floating point operations counted. This number is then scaled
up to the total amount of data that will need to be processed (since the
test only processes a small amount of) and then divided by the amount of
time in which the processing must take place (1 min or 6/12 months) to
produce the needed processing power in TFLOPS. The OTT1 test includes
the time allotted for processing the data, the actual measured overheads
incurred by the software and a realistic assessment of the amount of
time it takes to transfer the data from Chile to the US (NCSA has a
simulator for this).

\textbf{Verification Code:} For the AP and DRP computational budgets the
AP and DRP pipelines need to be run on a decent amount of simulated LSST
data while keeping track of the total number of floating pointing
operations and the overhead for the case of the AP. For the OTT1 test
the AP overheads are used as well as the NCSA data transfer simulator to
give a realistic transfer time.

\textbf{Verification Implementation:} TBD

\textbf{Verification Data:} Simulated full-frame LSST data will be
needed for both the AP and DRP processing. For AP a number of handful of
visits will be required. For DRP enough visits to realistically model
coadds for DR1 will be needed for a small number of fields.

\subsection{Database Metrics}\label{database-metrics}

\textbf{Reference Document:} \citell{LSE-81}\textbar{}G270, G283, G284. Also see
\citell{LDM-135}.

\textbf{Discussion:} These are metrics on the performance of the Qserv
database service. See \citell{LDM-135} section 4.3 for more information on ``low
volume'' and ``high volume'' queries.

\begin{longtable}[]{@{}p{0.4\textwidth}llp{1.5in}p{0.75in}@{}}
\toprule
Description & Value & Unit & Name & JIRA ID\tabularnewline
\midrule
\endhead
Qserv query rate & 150 & Simult. queries & query rate &
DLP-659\tabularnewline
Qserv low volume query response time (Type A-D) & 10 & seconds & LV
query response time & DLP-651 -- DLP-654\tabularnewline
Qserv low volume Object/Source joins response time & 12 & hours & LV
Object/Source joins response time & DLP-657\tabularnewline
Qserv high volume object scan response time & 1 & hours & HV object scan
response time & DLP-655\tabularnewline
Qserv high volume ObjectExtra scan response time & 8 & hours & HV
ObjectExtra scan reponse time & DLP-656\tabularnewline
Qserv high volume Object/ForcedSource join response time & 12 & hours &
HV Object/ForcedSource join reponse time & DLP-658\tabularnewline
\bottomrule
\end{longtable}

\textbf{Verification Owner:} NCSA

\textbf{Verification Technique:} The Qserv database will be loaded with
a large precursor dataset. Sample queries will be run at a high rate and
the response time measured. The results will need to be scaled up to the
size of the eventual LSST database.

\textbf{Verification Code:} Software will need to written to
auto-generate a large number of queries of various types, submit the
queries to Qserv, receive the returned data and measure the response and
delay time.

\textbf{Verification Implementation:} TBD

\textbf{Verification Data:} For this test a precursor dataset is needed
that is large enough to allow the results to be appropriately scaled up
to the LSST database size. The Pan-STARRS1, SDSS or DES survey datasets
should be good for this purpose.

\subsection{Web UI Metrics}\label{web-ui-metrics}

\textbf{Reference Document:} JIRA tickets.

\textbf{Discussion:} These are requirements on the speed of the web user
interface (UI).

\begin{longtable}[]{@{}p{0.4\textwidth}llp{1.5in}l@{}}
\toprule
Description & Value & Unit & Name & JIRA ID\tabularnewline
\midrule
\endhead
Web UI portal can support concurrent users. & 100 & users & Web UI
portal concurrent user supported & DLP-330\tabularnewline
Web UI query results display response time after the search results have
been received from the data provider. & 1 & seconds & Web UI query
response time & DLP-331\tabularnewline
Web UI image preparation (convert to PNG or other image format) time on
the server side. & 0.5 & seconds & Web UI image preparation time &
DLP-332\tabularnewline
Web UI image rendering time on user's display. & 0.5 & seconds & Web UI
image display rendering time & DLP-333\tabularnewline
\bottomrule
\end{longtable}

\textbf{Verification Owner:} SQuaRE

\textbf{Verification Technique:} Precursor datasets will be loaded into
a database and queried via the web UI and the response and rendering
times measured (on the server or user side as appropriate). Software
will need to be used to emulate many users for the first metric.

\textbf{Verification Code:} The server side and user side software will
need to be able to measure the rendering and response times to test
these metrics. Software to emulate \textasciitilde{}100 users, including
simulated queries and the capture of the response, will be needed used
or developed for this test.

\textbf{Verification Implementation:} TBD

\textbf{Verification Data:} Precursor datasets such as WISE or
Pan-STARRS1 can be used for this test.

\section{System Reliability Metrics}\label{system-reliability-metrics}

There are additional requirements on the reliability of the entire DM
system.

\subsection{Science Visit Alert Generation
Reliability}\label{science-visit-alert-generation-reliability}

Reference Document: \citell{LSE-30}; OSS-REQ-0112

\textbf{Discussion:} No more than sciVisitAlertFailure \% of science
visits read out in the camera {[}and specified to be analyzed by Data
Management{]} shall fail to be subjected to alert generation and
distribution, integrated over all stages of data handling from data
acquisition through transmission of the alerts across the project
boundary. No more than sciVisitAlertDelay \% of science visits read out
in the camera {[}and specified to be analyzed by Data Management{]}
shall have their alert generation and distribution completed later than
the \SRD specification for alert latency (OTT1).

The "specified to be analyzed" language allows for the possibility that
under some operating modes, such as diagnostics, images deliberately not
be analyzed for alerts.

This requirement applies to visits, and not to individual alerts,
because a specification that, e.g., "no more than 1\% of alerts shall
fail to be generated" gets tangled with questions of the scientific
performance of the actual alert detection. This requirement is a
performance specification on the DM system, taking the alert-detection
algorithm as a given.

\begin{longtable}[]{@{}p{0.4\textwidth}llll@{}}
\toprule
Description & Value & Unit & Name & JIRA ID\tabularnewline
\midrule
\endhead
Maximum fraction of science visits for which alerts are generated, but
delivered later than the OTT1 latency specification. & 1 & Percent &
sciVisitAlertDelay & -\/-\tabularnewline
Maximum fraction of visits for which alerts are not generated or
distributed. & 0.1 & Percent & sciVisitAlertFailure &
-\/-\tabularnewline
\bottomrule
\end{longtable}

\textbf{Verification Owner:} SQuaRE, with help from University of
Washington initially

\textbf{Verification Technique:} This requirement will need to be tested
with a ``full-up'' test of the DM system. This might not be possible
until commissioning with the full camera.

\textbf{Verification Code:} This metric can be evaluated by using the
internal database of when visits were taken and their alerts generated.

\textbf{Verification Implementation:} TBD

\textbf{Verification Data:} This metric will need to be tested with the
``full-up'' DM system, likely during commissioning with the full camera.

\subsection{Science Image Archiving
Reliability}\label{science-image-archiving-reliability}

Reference Document: \citell{LSE-30}; OSS-REQ-0111

\textbf{Discussion:} No more than sciImageLoss \% of science, wavefront,
and guider images read out in the camera and specified to be acquired by
Data Management shall be permanently lost or corrupted, integrated over
all stages of data handling through archiving and availability of the
image for access in the archive, or have its essential image acquisition
metadata permanently lost or disassociated with the image.

Corruption of images shall mean changes of any image pixel values, with
the exception of application of a compression algorithm. Any such
compression algorithm must be shown to have no effect on any \SRD
requirement.

The "specified to be acquired" language allows for the possibility that
under some operating modes, such as diagnostics, images might be
acquired by the camera and deliberately not archived.

\begin{longtable}[]{@{}p{0.4\textwidth}llll@{}}
\toprule
Description & Value & Unit & Name & JIRA ID\tabularnewline
\midrule
\endhead
Maximum fraction of read-out raw images permitted to be permanently lost
or corrupted downstream, including loss due to the loss or corruption of
essential associated metadata. & 1 & Percent & sciImageLoss &
-\/-\tabularnewline
\bottomrule
\end{longtable}

\textbf{Verification Owner:} NCSA

\textbf{Verification Technique:} This requirement will need to be tested
with a ``full-up'' test of the DM system. This might not be possible
until commissioning with the full camera.

\textbf{Verification Code:} This metric can be evaluated by using the
metadata database, the omission of any exposure numbers in the database,
and the quality assurance metrics on the corruption of data.

\textbf{Verification Implementation:} TBD

\textbf{Verification Data:} This metric will need to be tested with the
``full-up'' DM system, likely during commissioning with the full camera.

\bibliography{lsst}

\end{document}
